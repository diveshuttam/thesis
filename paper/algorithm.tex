\section{Algorithm and Analysis}

% How are we inspired from other related works
We propose a algorithm which blends the good features of the schemes which we referred in \ref{section:related_work} and improves upon many of their flaws at the same time. 

We are inspired from flow sense\cite{flowsense} in the feature that for small flows, we use the flow removed messages to measure the traffic. 
Inspired from OpenTM we poll only the edge switches to get the flow data. 
Derived from Payless and Multi-Objective we are more sensitive to increase in traffic than to the decrease.
Derived from CeMon we propose a parameter less approach to monitor any traffic.
We use some of the ideas from Multi-Objective to improve upon weakness of CeMon e.g. by using a small window as it depicts a better change.

% How are we better from each of the mentioned related work. Address just the different details.
Though we are similar to the related works, we improve upon quite a bit.
Unlike OpenTM we don't poll at a constant rate and use a dynamic frequency approach.
Unlike Payless our approach does not have parameter to be tuned.
Our approach has a better overhead and accuracy trade off than CeMon and Multi-Objective.
We use a curvature based approach which we show to be a better criteria for polling than other approaches which use change as a criteria.

% A detailed overview of our algorithm

Please see Algorithm \ref{algo:algorithm} for the pseudo-code.
\begin{algorithm}
    \DontPrintSemicolon
    \SetAlgoLined
    \SetKwInOut{Input}{Input}\SetKwInOut{Output}{Output}
    \SetKwInOut{Parameters}{Parameters}
    \Input{stat, stat\_time \tcp{Polled stat and time of polling}}
    \Parameters{$\Delta$ \tcp{measure of curvature}}
    \Output{Next polling time}
    \tcp{Set the initial values}
    polling\_interval = 1 \\
    max\_time $\leftarrow$ $m_{T}$  \tcp{$m_{T}$ is maximum polling interval}
    min\_time $\leftarrow$ $m_{t}$  \tcp{$m_{t}$ is minimum polling interval}
    
    window\_time $\leftarrow$ [] \tcp{time at which data was polled}
    window\_stat $\leftarrow$ [] \tcp{stats data}
    window\_dstat $\leftarrow$ [] \tcp{difference between two consecutive stats}
    
    % append to window
    window\_time.append(time) \\
    window\_stat.append(stat) \\
    window\_dstat.append((window\_stat[-1]-window\_stat[-2])/(window\_time[-1]-window\_time[-2])) \tcp{append rate of consecutive elements}
    
    % adjust the rate
    \tcp{only if sufficient entries in window}
    \If{$window\_stat.length>3$}{
    \tcp{check for curvature: if consecutive rate is close to the alternate rate}
    \If{$abs(windows\_dstat[-1]-window\_dstat[-2])<\Delta$}{$polling\_interval*=2$ \\
    \Else {$polling\_interval/=3$}
    }
    
    % new\_period $\leftarrow$ T/(2*change\_count) \\
    polling\_interval $\leftarrow$ $max(min(m_{T},polling\_interval),m_{t})$  \tcp{bring in range $m_{t}:m_{T}$} 
    }
    
    SetTimeout(PollingInterval, Curvature Based Sampling) \tcp{call the same function after the timeout}
    
    \caption{Curvature Based Sampling}
    \label{algo:algorithm}
    \end{algorithm}
    
Our algorithm is similar in structure to CeMon or Multi-Objective in the aspect that we use a multiplicative change in frequency when our algorithm detects a change.
The difference is that we use curvature as a criteria to detect change and use different factors than those algorithms.

% Why is a curvature a better criteria
Most schemes (Multi-Objective, CeMon etc.) uses the criteria "If change in measured property is more than some factor between two polls" then increase polling rate else decrease polling rate.
Instead of this we use the criteria "If significant curvature is detected (absolute difference between slopes within 3 polls is more than 500)" we increase the polling else decrease.
We increase the polling rate else we decrease the polling rate.
To see why our method is better than that of other schemes we look at the following curves.


% curve 1 linear line increasing a constant rate

In \ref{Fig 1} \ref{Fig 2} we have a curve which is a straight line.
Most other schemes either increase the polling rate or decrease it based on the slope of the line as they use the rate as a measure to check if there is a change.
The best strategy here is to reduce the polling rate as it will not affect accuracy at the same time reduce the overhead.

In \ref{Fig 3} \ref{Fig 4} \ref{Fig 5} we have curve which change in slope as shown in the points.
Increasing the polled points though increases the overhead it also decreases the error depending on the curvature as seen in figure \ref{fig 6}

The steps in our algorithm are as follows:
\begin{itemize}
    \item We maintain a window of last 3 polled stats and their slopes (slope1 between points 1 and 2, slope2 between points 2 and 3)
    \item We check if the slope1 and slope2 is more than a constant (500) if it is so we decrease the polling time by a factor of 3.
    \item Else if the difference is less than the constant, we increase the polling interval by a factor of 2.
    \item The polling interval is bounded between tmin(=0.5s) and tmax(=5.0s)
    \item We also have another novelty, since we decrease by a factor of 3 which is relatively high, we don't decrease on two continues polls. This helps in quick decrease while keeping the overhead a bit constrained. 
\end{itemize}

We use the constants 3 for decrease and 2 for increase as they are co-prime.
One of the problems with CeMon SWT \cite{cemon} was that it used the same constants (2) on both increase and decrease.
The problem with this is that the polling frequency is discretely bounded by the set {0.5, 0.625, 1, 1.25, 2, 2.5, 4, 5}s.
As we can see CeMon can't have a polling frequency between 2.5 and 4 seconds which causes error in polling traffic with variations of the order 3s.

Another problem with CeMon which we resolve is that we find through experiments that only recent windows play an active role in determining the rate of change.
CeMon suffers with the problem as we see in experiments whenever there is a sharp change in traffic, the standard deviation rises to a high value and the polling interval is maintained at tmax for a long time.
This strategy of less window size is also used by Multi-Objective. 
Though Multi-Objective also uses the rate change criteria unlike our algorithm in which we use curvature as the criteria.