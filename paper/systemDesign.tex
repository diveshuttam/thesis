\section{System Design}
\label{system_design}
\subsection{Background}
\label{background}
% openflow standard
In past few years OpenFlow \cite{openflow} has become de-facto open standard for communication between SDN controller and SDN switches. 
It is the most widely used standard for deployment of SDN.
The standard specifies protocol for communication between control and data plane.
OpenFlow standard allows the controller to manage flow entries in the switches to achieve routing and monitoring functionalities.
The switches maintains the count of number of packets and bytes matched to a particular flow entry or transferred through each port and each queue.
SDN controller can request these statistics using the statistics request messages provided by OpenFlow. 

% out of band vs in band
SDN can be deployed in two configurations, Out-of-Band or In-Band.
In the case of Out-of-Band deployment, the control messages are transmitted over a dedicated control link from SDN controller to the SDN switch.
Whereas, in In-Band configuration the SDN controller is connected to single switch in the network and both control and data traffic is transmitted over the same link called \textit{``data link"}. 
The Out-of-Band configuration has more cost than In-Band deployment in terms of hardware cost and management of the dedicated link between SDN controller and switches.
But in case of In-Band the data link is carrying whole traffic (control and network). Thus, there is an overhead on the data link in case of In-Band configuration.
There exists a trade-off between overhead on data link and hardware, and management cost \cite{in-band_and_out-band}.
Our approach works for both In-Band and Out-of-Band SDN deployments.

% Dependent variable:
% accuracy, polling overhead
% Independent variable:
% frequency, switches flow length etc.
% Assumptions: Distributions
% logical reasoning

% Curvature based sampling
\subsection{Sampling technique}
The problem of polling can be formulated as that of sampling of a continuous curve into discrete points.
Some popular sampling techniques for curves are highlighted in \cite{curvature}. These include:
\begin{enumerate}
\item
    \textbf{Arc length sampling} where the curve is sampled at equal distance on the arc.
\item
    \textbf{Curvature based sampling} where the sampling is done in proportion to the instantaneous curvature of the curve
\item
    \textbf{Mixed Sampling} where a combination of both Arc based sampling and Curvature based sampling is used.
\end{enumerate}

% Therefore, In-Band configuration is much more common in practice. Routing for the control messages in case of in-band deployment is determined by the network administrator. A separate VLAN can be configured for delivering control messages \cite{CEMON}. 

In this paper we use a modification of simple curvature based sampling for our case. 
Though we can't use the highlighted techniques as it is.
This is due to the fact that arc based sampling in a way assumes the whole curve to be available so as to find and divide the length of curve into equal parts.
It is not the case in our scenario. Also, curvature based sampling needs instantaneous curvature which is not available.
We can only approximate average curvature between polls.
To resolve this we modify the curvature based sampling to use average curvature as highlighted in the algorithm \ref{section:algorithm}.


% TODO: Architecture Figure
\begin{figure}[ht]
\includegraphics[width=.5\textwidth]{images/architecture.eps}

\caption{Curvature Based Sampling Architecture.}
\label{fig:Curvature Based Sampling_arch}
\end{figure}
